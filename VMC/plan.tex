\documentclass[12pt]{article}
\usepackage{helvet}
\usepackage{fullpage}
\newcommand{\bR}{{\bf R}}
\newcommand{\br}{{\bf r}}
\newcommand{\bP}{{\bf P}}
\begin{document}

\section{The system} 

We will implement variational Monte Carlo for the He atom. 

\subsection{Learning objectives}
\begin{itemize}
\item How to compute wave functions and expectation value.
\item How to sample many-body wave functions using the Metropolis algorithm.
\item The interplay between interactions and correlations. 
\end{itemize}

There are useful notes in each of the files that will tell you exactly what the inputs and outputs of each function should be.

\section{Implementing the pieces}

In order to implement a variational Monte Carlo (VMC) program, we need:
\begin{itemize}
	\item $\Psi(\bR,\bP)$, $\nabla \Psi(\bR,\bP)$, and $\nabla^2 \Psi(\bR,\bP)$ (slaterwf.py)
	\item A way to compute $\frac{H\Psi(\bR,\bP)}{\Psi(\bR,\bP)}$ (hamiltonian.py)
	\item A function to generate samples with probability proportional to $|\Psi|^2$ (metropolis.py)
\end{itemize}

Each of these files has a testing program built-in that you can use to evaluate your implementation.

\subsection{Wave function object (slaterwf.py)}

\begin{equation}
\Psi(\bR)=\exp(-\alpha r_1) \exp(-\alpha r_2)	
\end{equation}

The variational parameter is $\alpha$.

\subsubsection{Testing}

Numerical versus analytic gradients.

\subsection{Hamiltonian object (hamiltonian.py)}

\begin{equation}
\frac{\hat{H}\Psi(\bR)}{\Psi(\bR) } = -\frac{1}{2} \sum_i \frac{\nabla_i^2 \Psi(\bR)}{\Psi(\bR)} - \sum_i \frac{2}{r_i} + \frac{1}{r_{12}},
\end{equation}
where $r_{12}=|\br_1-\br_2|$ is the distance between the two electrons.

\subsubsection{Testing}

We have computed the potential energy for a few configurations by hand. 

\subsection{Metropolis algorithm (metropolis.py)}

We would like to draw $\bR$ from the distribution $\Psi^2(\bR)$. 
The Metropolis-Hastings algorithm does this by the following process:
\begin{enumerate}
\item Start with an initial configuration $\bR_0$. 	
\item Propose a new configuration $\bR'=\bR_0 + \sqrt{\tau} \chi$, where $\chi$ is a gaussian random number.
\item Compute the acceptance probability $a=\frac{\Psi^2(\bR')}{\Psi^2(\bR_0)}$
\item Generate a uniform random number $u$ between 0 and 1. 
\item If $u < a$, then set $\bR_1=\bR'$. Otherwise, set $\bR_1=\bR_0$.
\end{enumerate}

Useful functions: 
\begin{itemize}
\item np.random.randn()
\item np.random.random()
\item Conditional slices in numpy: R[:,:,$u<a$]=Rprime[:,:,$u<a$]
\end{itemize}


\subsubsection{Testing}

Exact wave function averages for non-interacting particles. We check kinetic, potential, and total energy.

\section{Optimizing one parameter}

At this point we have a system that can evaluate properties of a wave function. 

\begin{itemize}
\item Where is the minimum if we don't include electron-electron interaction? Notice anything about the errors at that point?
\item What happens to the minimum when interactions are included? Errors?
\item What is the behavior of the kinetic and potential energies as a function of $\alpha$? Do they make sense?	
\end{itemize}


\section{Add a Jastrow factor: optimizing two parameters} 

We have implemented a Jastrow wave function and an object MultiplyWF which can construct a Slater-Jastrow wave function.
You can construct it by doing
\begin{verbatim}
wf=MultiplyWF(SlaterWF(alpha),JastrowWF(beta))	
\end{verbatim}
The functional form of this simple Jastrow factor is 
\begin{equation}
\Psi_J(\bR) = \exp(\beta r_{12})	
\end{equation}

Optimize $\alpha$ with this new Jastrow. What happens to the optimal value? Why? 

\section{Computing expectation values}

\begin{itemize}
\item Radial distribution function	
\item Kinetic and potential energies. 
\end{itemize}

What happens to the average distance between electrons with the Jastrow factor? 

How does the Jastrow factor affect the kinetic and potential energies? 


\section{Improving the sampling: biased moves} 


\section{Excited states: triplet. } 

\end{document}
