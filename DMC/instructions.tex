\documentclass[12pt]{article}
\usepackage{helvet}
\usepackage{fullpage}
\usepackage{amsmath}
\begin{document}
	{\bf Diffusion Monte Carlo } 

\section*{The trajectories.} 

We're going to write a series of implementations of diffusion Monte Carlo.
Each implementation will generate data in the following form: 

\begin{tabular}{ccccc}
tau&step&config number&	local energy&weight\\
\hline
\end{tabular}


	
\section{Importance sampling} 

For 1000 walkers, generate dynamics as follows: 
\begin{equation}
x_{n+1} = x_n + \sqrt{\tau}\chi + \tau \frac{\nabla \Psi_T(x_n)}{\Psi_T(x_n)},
\end{equation}
where $\chi$ is a random variable. 
Set all weights to one.
Make a CSV file and use Pandas to analyze your data. 

\begin{itemize}
\item What is the limiting behavior as $\tau$ goes to zero? It should match the VMC result for that trial function.	
\end{itemize}

\section{Importance sampling with weights}
Now we will update the weights $w_i$. Set 
\begin{equation}
w*=	\exp[-\tau E_L(x_{n+1})] 
\end{equation}
each step.
You probably don't need to perform this for too many steps!

\begin{itemize}
\item Track the weights of the walkers as a function of step. What happens?
\end{itemize}

\section{Fixing the normalization}

We want the weights to average around 1.
Use a shift 
\begin{equation}
w*=	\exp[-\tau (E_L(x_{n+1}-E_{\text{ref}})] 
\end{equation}
We can adjust $E_{\text {ref}}$ to ensure the weights average to one:
\begin{equation}
E_{\text{ref}}=E_{\text{ref}}-\log(\langle w \rangle) 
\end{equation}
where $\langle w \rangle$ is the average weight. 

\begin{itemize}
\item Now plot the weights. They should average to 1 but what happens to the variance?
\end{itemize}

\section{Branching} 

We now want to split the walkers with too-large weights and kill the walkers with too-small weights. 
There are many ways to do this; we will choose one that lets us keep the number of walkers constant. 
Every step, 
{\bf XXX-fill this in} 

\section{Using the algorithm}
Congratulations! We have now implemented the DMC algorithm. 
Let's check some things:

\begin{itemize}
\item The exact energy is {\bf XXXXXX }. Do we get that in the $\tau\rightarrow0$ limit?
\end{itemize}

\section{More things to do}
\begin{itemize}
\item Add an acceptance/rejection step into the dynamical moves. What is the behavior of the average energy as a function of $\tau$? Can we improve the timestep error?	
\end{itemize}


\end{document}